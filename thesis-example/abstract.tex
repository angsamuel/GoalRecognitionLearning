%
% spacing could probably be improved
%

\begin{center}

\bigskip

\begin{Large}
\textbf{\theTitle}
\end{Large}

\bigskip

\begin{large}
\theAuthor
\end{large}

\bigskip
\bigskip

\textbf{Abstract}
Motivated by the goal recognition (GR) 
and goal recognition design (GRD) problems 
in the artificial intelligence (AI) planning domain, 
we introduce and study two natural variants of the 
GR and GRD problems with strategic agents, respectively.  
More specifically, we consider game-theoretic (GT)
scenarios where a malicious adversary 
is aiming to cause damage to some target in an
(physical or virtual) environment monitored by a defender.
The adversary is interested in attacking 
a single target and must take a sequence of actions in order to attack the target. 
In the GTGR and GTGRD settings, the defender's goal is to identify the adversary's 
intended target from observing the adversary's actions 
so that he/she can strengthens the target's defense against the attack. 
In addition, in the GTGRD setting, the defender can alter the environment (e.g., adding roadblocks) 
in order to better distinguish the goal/target of the adversary. 

%The intended target is adversary's private information 
%while the set of potential targets is known to both parties. 

%For instances, in physical security domains, this could be a sequence of 
%physical movements reaching the target; 
%in cyber security domains, this could be a sequence of 
%actions achieving necessary subgoals to carry out the attack. 

We propose to model GTGR and GRGRD settings as zero-sum stochastic games
with incomplete information about the adversary's intended target. %(i.e., private information).
The games are played on graphs where vertices are states and edges are adversary's actions.
For the GTGR setting,
we show that if the defender is restricted to playing only stationary strategies, 
the problem of computing optimal strategies (for both defender and adversary)
can be formulated and represented compactly as a linear program.
For the GTGRD setting,
where the defender can choose $K$ edges to block at the start of the game, 
we formulate the problem of computing optimal strategies as a mixed integer program,
and present a heuristic algorithm
based on LP duality and greedy methods.
Experiments show that our heuristic algorithm
achieves good performance (i.e., close to defender's optimal value) with better scalability
compared to the mixed-integer programming approach.

In contrast with our research, existing work, especially on GRD problems, have focused almost exclusively on decision-theoretic paradigms, where the adversary chooses its actions without taking into account the fact that they may be observed by the defender. As such an assumption is unrealistic in GT scenarios, our proposed models and algorithms fill a significant gap in the literature.
\end{center}

\noindent
% FIX THIS --- your abstract goes here.
A nice abstract goes here.

